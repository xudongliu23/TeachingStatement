\documentclass[12pt]{article}

\usepackage{setspace} 
\usepackage{times}
\usepackage{fullpage}

\singlespacing

\newcommand{\tit}[1]{\textit{#1}}
\newcommand{\tbf}[1]{\textbf{#1}}

\title{Teaching Statement}
\author{Xudong Liu}
\date{}

\begin{document}

\maketitle% prints the title block

Growing up, I was immersed in an environment committed to academic achievement
because my parents were teachers.
They often spoke proudly about their students who
struggled through tough days early on, overcame the hurdles with support of teachers, and 
achieved excellence inside and outside of school. 
Hoping to earn the same sense of satisfaction, I have always had a dream of becoming a teacher.
As I am finishing graduate school, where I have flourished in teaching,
I surely feel prepared to take the next step and teach at the college level.
The way I see it,
teaching is quintessential in helping people make a positive difference in their lives.

The ultimate goal of college teaching is to help students grow into independent 
and critical thinkers and problem solvers.
To work towards that goal, I believe that a college teacher should:
(1) spark and sustain students' interest in the discipline,
(2) provide them with an interactive environment for effective learning, and
(3) develop their abilities to identify, formalize,
and solve problems.

I see great value in hands-on learning, where the instructor offers fun
applications of the material whenever possible,
to foster students' interest in the current subject.
Designing and implementing a GUI that collects and reasons about user constraints and preferences
was one of the most inspiring projects in all my graduate classes.
Eventually, this project stimulated my research interests in answering theoretical and
practical questions related to preferences.
Hence, I plan to integrate teaching with research via exciting projects that
drive the students to discover more.
Furthermore, introducing hard concepts through intuitive examples often
gets the ``Wow" response from the students and
encourages them to dig deeper.
During a lecture for the course on artificial intelligence,
I was introducing the first-order logic and the generalized modus ponens
rule.  Noticing that some students had difficulty understanding this
important inference rule, I illustrated it with the 
Aristotle's syllogism and explained how unification works in the rule.
An acknowledgment from the audience ``Ah, that's what it is!" indicates
the significance of exemplification, and of connecting the unknown with
what is known, in teaching.

An effective teacher also needs to keep the students engaged
and unfold the material productively, as classes are typically not very long.
Over the years in my Ph.D. career, I have given various technical
talks at premier workshops and international conferences.
Presenting research results at these venues in front of general audiences
provides a good introduction to teaching a high level class in computer science,
in that both, in a small portion of time, define the problem at hand, discuss motivation and related work,
and show new methods and results.

To guarantee effective learning for the audience, 
I hold that a teacher should be committed to
creating an interactive and adaptive learning environment both inside
and outside the classroom.
Timely interaction between the teacher and the students in class can lead to improved
comprehension of the material just covered.
When a problem seems common among many students, the instructor should quickly
see it and address it.
For instance, when pointers were introduced in CS 215 on C++ programming, 
many of my students were confused when parsing pointers declared with the keyword \tit{const}.
Not only did I explain the effects of having the keyword in different places in the
declaration, I also came up with an in-class quiz for students to test whether there
will be a compile-time or a run-time error.
Having done the quiz and heard an explanation, the students became comfortable with
pointers and constant variables, and this was reflected in later programming assignments.

The instructor in charge of an inclusive class should
pay attention to subtle signals from the students, 
and adjust the rhythm of the course correspondingly.
Leading many lab sessions for programming classes,
I noticed there were several shy students who would not approach me with questions,
which sometimes negatively reflected in their lab work.
To help these students thrive and form a effective learning environment for everyone, 
I interacted with all students to make sure everyone had made progress.
Moreover, the teacher should make efforts, even offline, 
to assure that students with legitimate requests are answered.
Whenever there were students who needed to but could not
meet me in my office, mostly due to schedule conflicts, 
I would always make an effort to talk to them outside of office hours and
to make certain their issues were solved.

To prepare students to become efficient problem solvers, 
I believe that a teacher should promote
constructive discussions where problems are identified and possible approaches are suggested.
For instance, in a flipped classroom, every student acts as a teacher by presenting papers
or demonstrating implementations.  During the event, the mentor observes the student and
provides helpful feedback, ensuring the student knows if there are problems and, if so,
how they could possibly be addressed.
I view constructive discussions as the key element of providing mentorship.
A college teacher, in my opinion, is always a learning researcher, staying
active with the advances in the field.
Amongst scientific disciplines,
computer science has perhaps one of the most dynamic landscapes, in the sense that
frontiers are pushed forward almost on an annual basis.
As a consequence, I see it is necessary to keep up-to-date in my area,
because it will help me remain highly motivated and well prepared as a mentor.
Reflecting upon my own education, I have been very fortunate to have mentors
that fortified my abilities to become a computer scientist.
I would love to do likewise.

From my teaching experience, I am confident that I can competently teach courses such as
\tit{programming languages}, \tit{data structures and algorithms}, \tit{software engineering},
\tit{logics and discrete mathematics},
\tit{computer networks}, \tit{database systems}, and \tit{operating systems}.
Based on my research training, I am comfortable teaching introductory and high 
level courses in \tit{artificial intelligence},
\tit{preferences and social choice}, and \tit{knowledge representation and reasoning}.
In particular, I intend to bring research into classrooms and teach courses such as
\tit{preference languages and theory}, \tit{social networks},
\tit{preference learning}, and \tit{preference handling}.


\end{document}
